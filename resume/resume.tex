% Created 2018-01-14 Sun 13:58
\documentclass[12pt letterpaper notitlepage]{article}
\usepackage[utf8]{inputenc}
\usepackage[T1]{fontenc}
\usepackage{fixltx2e}
\usepackage{graphicx}
\usepackage{longtable}
\usepackage{float}
\usepackage{wrapfig}
\usepackage{rotating}
\usepackage[normalem]{ulem}
\usepackage{amsmath}
\usepackage{textcomp}
\usepackage{marvosym}
\usepackage{wasysym}
\usepackage{amssymb}
\usepackage{hyperref}
\tolerance=1000
\pagenumbering{gobble}
\usepackage{helvet}
\renewcommand{\familydefault}{phv}
\usepackage{parskip}
\usepackage[margin=0.5in]{geometry}
\date{}
\title{Aaron Kuehler - Resume}
\hypersetup{
  pdfkeywords={},
  pdfsubject={Professional information of Aaron Kuehler},
  pdfcreator={Emacs 24.5.1 (Org mode 8.2.10)}}
\begin{document}

\maketitle
\begin{quote}
Always remember, our purpose is not to "write code", but rather add
value to the businesses.
\end{quote}

\section*{Skills / Experience Summary}
\label{sec-1}

\begin{itemize}
\item Web Application Development (Ruby/Rails, Elixir/Phoenix, and Standard Web Technologies - JavaScript, CSS, etc)
\item API Development (Ruby/Rails, Elixir/Phoenix)
\item Continuous Integration and Deployment Automation
\item Team Growth and Leadership through agility, Lean Manufacturing Principles, and Extreme Ownership
\item Individual Mentor / Mentee
\end{itemize}

\section*{Experience}
\label{sec-2}

\subsection*{Bauer Xcel Media}
\label{sec-2-1}

\textbf{Technical Team Lead / Senior Software Developer}

\emph{February 2015 - Present}

At Bauer, Technical Team Lead is a role which augments a member of the
development staff's role slightly. In addition to contributing to the
development process, they are required to act as force multipliers. They are
responsible for the augmentation of team effectiveness and efficiency. They
inspire change in individual and team behavior to achieve this goal.

During my time as team lead we've reduced average Story cycle time by
34.37\%. We've used pair programming as a means to build trust and rapport
between teammates - removing waste in unnecessary review processes and improving
morale. We've trimmed waste and refocused the team to reduce WIP and plan more
appropriately to the constraints of our pipeline. We are able to react to
unplanned work with minimal risk to commitments. We more predictably delivered
on commitments which has built trust with stakeholders - planning and
negotiations run more smoothly as a result.

During an internal "Hack-a-thon" in 2017, I rallied a team of 3
developers behind the idea of deployment automation. In two days we
completed work which made on-demand application review instances -
review instance per pull request - and continuous delivery to
production a reality at Bauer. The work-products of these two days
alone saved the company 50\% in month-over-month hosting costs.

\subsection*{Sizmek (formerly PointRoll)}
\label{sec-2-2}

\textbf{Software Developer}

\emph{January 2015 - June 2016}

In January of 2015, Pointroll was in the midst of a turbulent reorganization. I
was brought on to shared my experience and ideas on becoming a more agile team;
changing the mind-set away from delivering code to delivering value to the
business. I helped shape development workflows and deployment pipeline to help
facilitate the concurrent development by varied teams across a multitude of code
bases. I helped create, contributed to, and ran the PointRoll Community or
Practice. A bi-weekly lunch-and-learn type talk series designed to distribute
knowledge and increase cross-team interpersonal relationships. Pointroll was
acquired for its customer base by Sizmek in November of 2015 for \$20 million. I
was part of a skeleton crew who "kept the lights on" until June 2016 when the
PointRoll systems were retired.

\subsection*{Hoopla}
\label{sec-2-3}

\textbf{Software Developer}

\emph{July 2012 - January 2015}

In July 2012 I became the 3rd full-time developer hire at Hoopla. I was brought
on to augment the development capacity of the team as the product grew into
larger markets and acquired bigger clients like: LinkedIn, Zillow, Angie's List,
and Ring Central. I would spend most of my first 2 years working on the core web
application and external event processing integration with Salesforce. Later I
would lead the development effort of the Native mobile version of the Hoopla
platform.

\subsection*{Artisan (formerly AppRenaissance)}
\label{sec-2-4}

\textbf{Senior Software Developer}

\emph{July 2011 - July 2012}

As the 3rd employee of, then, AppRenaissance I helped grow a small
mobile-developers-for-hire into a small mobile products/platform company. I
spent most of my time helping clients with product and business
development. Later we would start to change focus to providing services and
products for mobile developers.

\subsection*{Infor}
\label{sec-2-5}

\textbf{Software Engineer}

\emph{May 2009 - July 2011}

In May 2009, Infor was in the midst of re-architecture of its ERP and logistics
management software. This re-architecture saw an old monolithic, database driven
application transformed into nearly two dozen independent, distributed
services. As the member of many teams within the team, I adied in the definition
and implementation of several of these components - from Requisition and Orders
to Accounts Payable modules. Early on I would establish the "Brown Bag Club"; an
opt-in lunchtime discussion forum and knowledge sharing opportunity.

\subsection*{Oracle (formerly AdminServer)}
\label{sec-2-6}

\textbf{Application Engineer}

\emph{February 2006 - June 2009}

Initially worked on the flagship Life \& Annuity insurance policy administration
system. Eventually I was asked to join the technology skunk-works and
performance teams to lay the foundation for the next-generation of this
product. Later on I would be asked to lead the "backend" development of the
next-generation of the policy administration system.

\section*{Open Source Contributions}
\label{sec-3}

\subsection*{heroku-cli-buildpack}
\label{sec-3-1}

\url{https://github.com/Thermondo/heroku-cli-buildpack}

Installs the Heroku toolbelt on a heroku dyno

\subsection*{keyword\_parameter\_matchers}
\label{sec-3-2}

\url{https://github.com/terryfinn/keyword_parameter_matchers}

RSpec matchers for method keyword parameters.

\subsection*{githug}
\label{sec-3-3}

\url{https://github.com/Gazler/githug}

Githug is designed to give you a practical way of learning git. It has a series
of levels, each requiring you to use git commands to arrive at a correct answer.

\section*{Projects}
\label{sec-4}

\subsection*{futurist}
\label{sec-4-1}

\url{https://github.com/indiebrain/futurist}

An implementation of the future construct, inspired by Celluloid's block based
futures, which uses process forking as a means of backgrounding work.

\subsection*{backbone-elasticsearch}
\label{sec-4-2}

\url{https://github.com/indiebrain/backbone-elasticsearch}

Adapters and Utilities to interface Backbone.js with ElasticSearch

\subsection*{OmniAuth Doximity OAuth2}
\label{sec-4-3}

\url{https://github.com/indiebrain/omniauth-doximity_oauth2}

An OmniAuth (\url{https://github.com/intridea/omniauth}) OAuth2 strategy for
Doximity (\url{http://www.doximity.com/})

\section*{Talks}
\label{sec-5}

\subsection*{Git Internals}
\label{sec-5-1}

\url{https://github.com/indiebrain/talks/blob/master/git_internals/git_internals.org}

Does git's user interface seem cryptic? Are you often confused about when you
should use 'checkout' vs 'reset'? Does 'rebase' feel scary? This talk explains
the inner workings of git and sheds a bit of light on how the internal structure
of git as a data store influences its user interface.

\section*{Elsewhere}
\label{sec-6}

\begin{itemize}
\item \url{http://www.aaronkuehler.com}
\item \url{http://www.github.com/indiebrain}
\item \url{http://twitter.com/indiebrain}
\end{itemize}

\section*{Education}
\label{sec-7}

\subsection*{West Chester University of Pennsylvania}
\label{sec-7-1}

\textbf{Bachelor of Science, Computer Science}
\textbf{Informantion Assurance Minor}

\emph{January 2006}

Graduating Magna Cum Laude, I achieved the Dean's list in 2005 and 2006, was
awarded the Honor of Academic Excellence in 2006.

\section*{Research}
\label{sec-8}

\subsection*{Small File Affects on Hadoop Distributed File System}
\label{sec-8-1}

\begin{itemize}
\item White Paper - \url{https://www.slideshare.net/slideshow/embed_code/key/S4XYiY0a4mOn8}
\item Presentation - \url{https://www.slideshare.net/slideshow/embed_code/key/9oo7oSckHxMTHX}
\end{itemize}

The Hadoop Distributed File System is a high throughput distributed File system
designed to accommodate large data sets; average file sizes in the
gigabyte-terabyte range. However when a data set is composed of large amounts of
small files, say in the kilobyte range, the storage system's semantics introduce
hight amounts of overhead in terms of file system block storage and read
latency. This paper explains the architectural attributes which cause these
problems and examines techniques to mitigate their impact when working with data
sets comprised of large numbers of small files.
% Emacs 24.5.1 (Org mode 8.2.10)
\end{document}
